{\bf Informe de Idoneidad para co-dirección} % O quizá Section

El ``Acuerdo del Comité de Dirección de la Escuela de Doctorado de Ciencias, Tecnologías e Ingenierías en relación con la Asignación de Director de tesis doctoral'' de la Universidad de Granada \footnote{\url{http://escuelaposgrado.ugr.es/doctorado/escuelasdoctorado/escueladoctoradocienciasingenieria/_doc/acuerdo_asignacion_director/!}} afirma lo siguiente:

{\em Como norma general, la dirección debe recaer en una sola persona salvo casos
debidamente justificados. Toda propuesta de codirección deberá estar justificada por
razones de índole académica como pueden ser el caso de la interdisciplinariedad
temática, los programas de doctorado desarrollados en colaboración nacional o
internacional o cuando uno de los directores sea novel.}

En esta tesis se propone la co-dirección del Doctor Pablo García
Sánchez. Las razones de esta decisión vienen dadas por el carácter
multidisciplinar de la misma, al combinar el uso de algoritmos de
Machine Learning con distintas tecnologías usadas ampliamente en el
mundo empresarial. El susodicho doctor, en su tesis doctoral y en
trabajos desarrollados a partir de la misma, ha combinado conceptos de
arquitecturas avanzadas de computación con algoritmos bioinspirados;
esta experiencia será sin duda útil en la tesis.

La experiencia de supervisión de Pablo García Sánchez viene avalada
por 2 co-direcciones de Trabajos Fin de Master y 5 Trabajos Fin de
Grado/Proyectos Fin de Carrera. Además, es Investigador Principal de
un proyecto de investigación dentro del marco de ayudas a la
investigación de la DGT, % código
 uno de cuyos paquetes utiliza varias de las
tecnologías anteriormente descritas. 
%Finalmente, al no haber
%co-dirigido ninguna tesis, se considera novel, justificando el
%apartado anteriormente mencionado.  
% Eso no *justifica* que puedas dirigirlo. Lo justificaría si *yo*
% fuera novel y tú fueras uno de fuera que fueras un crack.

