\documentclass{beamer}
\usepackage[T1]{fontenc}
\usepackage[utf8]{inputenc}
\usepackage{lmodern}
\usepackage[spanish]{babel}

\mode<presentation>
{
  \usetheme{Ilmenau} 
  \usecolortheme{dolphin}
  \usefonttheme{serif} 
  \setbeamertemplate{navigation symbols}{}
  \setbeamertemplate{caption}[numbered]
} 


%%%%%%%%%%%%%%%%%%%%%%%%%%%%%%%%%%%%%%%%%%%%%%%%



\title[Plan de Investigación]{Plan de Investigación para la Tesis Doctoral}
\author{Paloma de las Cuevas Delgado}
\logo{\includegraphics[scale=0.75]{imgs/logo_ugr.png}}
\institute{Departamento de Arquitectura y Tecnología de los Computadores}
\date{\today}

\begin{document}

\begin{frame}
  \titlepage
\end{frame}

\begin{frame}{Índice}
  \tableofcontents
\end{frame}

\section{Introducción}

\begin{frame}{Introducción}

\begin{itemize}
  \item Título provisional
\end{itemize}

\end{frame}

\section{Hipótesis}

\begin{frame}{Hipótesis}

Aquí se puede escribir.

\begin{itemize}
  \item Título provisional
\end{itemize}

\end{frame}

\section{Estado del arte}

\begin{frame}{Estado del arte}

Se pueden poner referencias a pie\footnote{Parece que sí}

\begin{itemize}
  \item IoT ?
\end{itemize}

\end{frame}

\section{Justificación}

\begin{frame}{Justificación}

\begin{itemize}
  \item TFM
  \item MUSES
\end{itemize}

\end{frame}


\section{Objetivos}

\begin{frame}{Objetivos}

\begin{block}{Primer objetivo}
Caracterizar una serie de entornos, o casos de uso, a contemplar para la posterior elección de métricas, técnicas, y forma de abordar los problemas.
\end{block}

Tareas del primer objetivo:

\begin{description}
  \item[1.a] Identificar los entornos que se quieren cubrir y los problemas a resolver en cada uno.
  \item[1.b] Para cada entorno identificado, identificar también el contexto y los datos que se requieren para el análisis de cada uno.
\end{description}

\end{frame}

\begin{frame}{Objetivos}

\begin{block}{Segundo objetivo}
El segundo objetivo de la tesis es definir un conjunto de métricas en referencia a la detección de ataques de seguridad producidos por la insuficiencia o ineficacia de un conjunto de políticas establecido.
\end{block}

Tareas del segundo objetivo:

\begin{description}
  \item[2.a] Realizar un estudio sobre el estado del arte
    relacionado con los entornos identificados. 
  \item[2.b] Comparativa de métricas usadas en la literatura: usos, ventajas, e inconvenientes de cada una.
  \item[2.c] Definición de la métrica y justificación de la misma.
\end{description}

\end{frame}

\begin{frame}{Objetivos}

\begin{block}{Tercer objetivo}
Elegir qué técnicas de Soft Computing y de Machine Learning resultan las mejores para cada uno.
\end{block}

Tareas del tercer objetivo:

\begin{description}
  \begin{small}
  \item[3.a] Estudio del arte en técnicas Soft Computing y Machine Learning aplicadas a los entornos identificados. Este sub objetivo está relacionado con el primer objetivo y la tarea 2.a.
  \item[3.b] Relación entre la métrica escogida o definida en el segundo objetivo con las técnicas estudiadas en la tarea 3.a.
  \item[3.c] Justificación de las técnicas escogidas respecto a la métrica escogida o definida en el segundo objetivo.
  \end{small}
\end{description}

\end{frame}

\begin{frame}{Objetivos}

\begin{block}{Cuarto objetivo}
Proponer una metodología que defina qué datos procesar, y cómo procesarlos, para obtener la información necesaria para adaptar las reglas de seguridad existentes.
\end{block}

Tareas del cuarto objetivo:

\begin{description}
  \begin{small}
  \item[4.a] Estudio de las técnicas de pre-procesamiento y elección de una o varias, justificando el por qué. Se requiere este objetivo para el segundo paso de la metodología.
  \item[4.b] Creación de un \textit{benchmark} para la experimentación a partir de datos reales. Este objetivo afecta a los pasos 1 y 2 de la metodología.
  \end{small}
\end{description}

\end{frame}

\begin{frame}{Objetivos}

\begin{block}{Cuarto objetivo}
Proponer una metodología que defina qué datos procesar, y cómo procesarlos, para obtener la información necesaria para adaptar las reglas de seguridad existentes.
\end{block}

Tareas del cuarto objetivo:

\begin{description}
  \begin{small}
  \item[4.c] Diseñar y definir el entorno de procesamiento con las técnicas escogidas en el segundo objetivo de la tesis. Este sub objetivo es necesario para definir el segundo paso de la metodología.
  \item[4.d] Definir cómo se validan esos resultados: comparando las métricas de cada técnica entre sí, los porcentajes de acierto en clasificación, y extrayendo conocimiento a partir de los datos.
  \end{small}
\end{description}

\end{frame}

\begin{frame}{Metodología propuesta}
La metodología estaría compuesta de varios pasos que irían detallados con las conclusiones sacadas del cumplimiento de los objetivos anteriores:

\begin{enumerate}
  \item Un adecuado pre-procesamiento de datos para asegurar un buen mantenimiento de las bases de datos y máximo rendimiento. Descripción de cómo preparar los datos antes de analizarlos.
  \item Entorno de ejecución completo, es decir, arquitectura y técnicas usadas.
  \item Validación de los resultados con técnicas propuestas, modo de uso e interpretación.
\end{enumerate}

\end{frame}

\section{Metodología}

\begin{frame}{Metodología}

\end{frame}

\section{Plan de trabajo}

\begin{frame}{Plan de trabajo}
Gantt
\end{frame}

\subsection{Medios y financiación}

\begin{frame}{Medios y financiación}
\$\$\$\$
\end{frame}

\end{document}
