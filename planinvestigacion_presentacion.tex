\documentclass{beamer}
\usepackage[T1]{fontenc}
\usepackage[utf8]{inputenc}
\usepackage{lmodern}
\usepackage[spanish]{babel}

\mode<presentation>
{
  \usetheme{Ilmenau} 
  \usecolortheme{dolphin}
  \usefonttheme{serif} 
  \setbeamertemplate{navigation symbols}{}
  \setbeamertemplate{caption}[numbered]
} 


%%%%%%%%%%%%%%%%%%%%%%%%%%%%%%%%%%%%%%%%%%%%%%%%



\title[Plan de Investigación]{Plan de Investigación para la Tesis Doctoral}
\author{Paloma de las Cuevas Delgado}
\logo{\includegraphics[scale=0.75]{imgs/logo_ugr.png}}
\institute{Departamento de Arquitectura y Tecnología de los Computadores}
\date{\today}

\begin{document}

\begin{frame}
  \titlepage
\end{frame}

\begin{frame}{Índice}
  \tableofcontents
\end{frame}

\section{Introducción}

\begin{frame}{Introducción}

\begin{itemize}
  \item Título provisional
\end{itemize}

\end{frame}

\section{Hipótesis}

\begin{frame}{Hipótesis}

Aquí se puede escribir.

\begin{itemize}
  \item Título provisional
\end{itemize}

\end{frame}

\section{Estado del arte}

\begin{frame}{Estado del arte}

Se pueden poner referencias a pie\footnote{Parece que sí}

\begin{itemize}
  \item IoT ?
\end{itemize}

\end{frame}

\section{Justificación}

\begin{frame}{Justificación}

\begin{itemize}
  \item TFM
  \item MUSES
\end{itemize}

\end{frame}


\section{Objetivos}

\begin{frame}{Objetivos}

\begin{block}{Primer objetivo}
Caracterizar una serie de entornos, o casos de uso, a contemplar para la posterior elección de métricas, técnicas, y forma de abordar los problemas.
\end{block}

\begin{block}{Segundo objetivo}
El segundo objetivo de la tesis es definir un conjunto de métricas en referencia a la detección de ataques de seguridad producidos por la insuficiencia o ineficacia de un conjunto de políticas establecido.
\end{block}

\end{frame}

\begin{frame}{Objetivos}

\begin{block}{Tercer objetivo}
Elegir qué técnicas de Soft Computing y de Machine Learning resultan las mejores para cada uno.
\end{block}

\begin{block}{Cuarto objetivo}
Proponer una metodología que defina qué datos procesar, y cómo procesarlos, para obtener la información necesaria para adaptar las reglas de seguridad existentes.
\end{block}

\end{frame}

\subsection{Tareas del primer objetivo}

\begin{frame}{Tareas del primer objetivo}

\end{frame}

\subsection{Tareas del segundo objetivo}

\begin{frame}{Tareas del segundo objetivo}

\end{frame}

\subsection{Tareas del tercer objetivo}

\begin{frame}{Tareas del tercer objetivo}

\end{frame}

\subsection{Tareas del cuarto objetivo}

\begin{frame}{Tareas del cuarto objetivo}

\end{frame}

\section{Plan de trabajo}

\begin{frame}{Plan de trabajo}
Gantt
\end{frame}

\subsection{Medios y financiación}

\begin{frame}{Medios y financiación}
\$\$\$\$
\end{frame}

\end{document}
