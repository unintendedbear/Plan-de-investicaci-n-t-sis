\documentclass{beamer}
\usepackage[T1]{fontenc}
\usepackage[utf8]{inputenc}
\usepackage{lmodern}
\usepackage[spanish]{babel}

\mode<presentation>
{
  \usetheme{Ilmenau} 
  \usecolortheme{dolphin}
  \usefonttheme{serif} 
  \setbeamertemplate{navigation symbols}{}
  \setbeamertemplate{caption}[numbered]
} 


%%%%%%%%%%%%%%%%%%%%%%%%%%%%%%%%%%%%%%%%%%%%%%%%



\title[Plan de Investigación]{Plan de Investigación para la Tesis Doctoral}
\author{Paloma de las Cuevas Delgado}
\logo{\includegraphics[scale=0.75]{imgs/logo_ugr.png}}
\institute{Departamento de Arquitectura y Tecnología de los Computadores}
\date{\today}

\begin{document}

\begin{frame}
  \titlepage
\end{frame}

\begin{frame}{Índice}
  \tableofcontents
\end{frame}

\section{Introducción}

\begin{frame}{Introducción}

\begin{itemize}
  \item Título provisional ``Soft computing techniques applied to corporate and personal security''
\end{itemize}

\end{frame}

\section{Hipótesis}

\begin{frame}{Hipótesis}

\begin{block}{}
Big Data, Internet de las Cosas, Bring Your Own Device, La Nube, Redes Sociales
\end{block}

\begin{itemize}
  \item Términos relacionados con la inserción de dispositivos inteligentes en la vida cotidiana.
  \item Relacionados también con la posibilidad de obtener datos del \textbf{contexto} en el que se encuentran.
  \begin{block}{Contexto}
    \textit{Cualquier información que pueda usarse para caracterizar la situación de una entidad.}\footnote{{\scriptsize Abowd, G. D., Dey, A. K., Brown, P. J., Davies, N., Smith, M., \& Steggles, P. (1999). \textit{Towards a better understanding of context and context-awareness. In Handheld and ubiquitous computing} (Springer)}}
  \end{block}
\end{itemize}

\end{frame}

\begin{frame}{Hipótesis}

\begin{itemize}
  \item Las amenazas evolucionan al mismo tiempo\footnote{\scriptsize Gangula, A., Ansari, S., \& Gondhalekar, M. (2013, November). \textit{Survey on Mobile Computing Security}. In Modelling Symposium (EMS), 2013 European (pp. 536-542). IEEE.}.
  \item Sistemas de seguridad actuales: aplican unas reglas predefinidas.
  \item ¿Qué ocurre con las amenazas no conocidas?
\end{itemize}

\end{frame}

\begin{frame}{Hipótesis}

\begin{block}{Hipótesis}
Demostrar que con un adecuado procesamiento de la información recopilada acerca del contexto que rodea a un usuario interaccionando con un dispositivo, además del análisis del comportamiento en el pasado, es posible adaptar el conjunto de reglas de seguridad existente y detectar de manera más precisa futuras amenazas.
\end{block}

\end{frame}

\section{Estado del arte}

\begin{frame}{Estado del arte}

Se pueden poner referencias a pie\footnote{Parece que sí}

\begin{itemize}
  \item IoT ?
\end{itemize}

\end{frame}

\section{Justificación}

\begin{frame}{Justificación}

Trabajo realizado hasta ahora:

\begin{itemize}
  \item Trabajo Fin de Máster. Estudio sobre la evolución de listas negras y blancas de URLs\footnote{Mora, A. M., De las Cuevas, P., \& Merelo, J. J. \textit{Going a Step Beyond the Black and White Lists for URL Accesses in the Enterprise by means of Categorical Classifiers}. ECTA 2014}.
  \begin{itemize}
    \item Reglas del tipo SI \textit{(url = xxxx)} ENTONCES \textit{[denegar, permitir]}
    \item Objetivo: obtener reglas que dependan de otros factores, o atributos, de la conexión HTTP.
    \item Atributos numéricos y categóricos
    \item Comparación de distintas técnicas de clasificación y de distintas configuraciones de porcentajes de entrenamiento y test.
    \item Resultados
  \end{itemize}
\end{itemize}

\end{frame}

\begin{frame}[fragile]{Justificación}

Trabajo realizado hasta ahora:

\begin{itemize}
  \item Trabajo Fin de Máster. Estudio sobre la evolución de listas negras y blancas de URLs\footnote{{\scriptsize Mora, A. M., De las Cuevas, P., \& Merelo, J. J. \textit{Going a Step Beyond the Black and White Lists for URL Accesses in the Enterprise by means of Categorical Classifiers}. ECTA 2014}}.
  \begin{itemize}
    \item Resultados
  \end{itemize}
\end{itemize}

\begin{columns}
  \begin{column}{.5\textwidth}
\begin{tiny}
\begin{verbatim}
(*)IF http_reply_code = "200"
   AND content_type = "application/json"
   AND time <= 33635000
   AND bytes <= 3921
   THEN ALLOW

(*)IF content_type = "text/plain"
   AND duration_milliseconds >= 7233.5
   THEN DENY
\end{verbatim}
\end{tiny}
  \end{column}
  \begin{column}{.5\textwidth}
\begin{tiny}
\begin{verbatim}
(*)IF content_type = "application/octet-stream"
   AND bytes <= 803
   THEN ALLOW

(*)IF bytes <= 1220
   AND time <= 33841000
   AND http_reply_code = "404"
   AND squid_hierarchy = DEFAULT_PARENT
   AND duration_milliseconds <= 233
   AND bytes <= 722
   THEN ALLOW
\end{verbatim}
\end{tiny}
  \end{column}
\end{columns}

\end{frame}

\begin{frame}{Justificación}

Trabajo realizado hasta ahora:

\begin{itemize}
  \item Escenario BYOD estudiado en el proyecto MUSES.
  \item Estudio del estado del arte en herramientas que abordan la situación BYOD.\footnote{{\scriptsize de las Cuevas, P., Mora, A. M., Merelo, J. J., Castillo, P. A., García-Sánchez, P., \& Fernández-Ares, A. (2015). \textit{Corporate security solutions for BYOD: A novel user-centric and self-adaptive system}. Revista Computer Communications (Elsevier).}}
  \begin{itemize}
    \item Es notable la falta de un sistema que evolucione las reglas de seguridad y no sólo las aplique.
  \end{itemize}
  \item Dataset de eventos producidos por los empleados de una empresa usando MUSES para BYOD en sus dispositivos.
  \begin{itemize}
    \item Incluídos datos de contexto.
  \end{itemize}
\end{itemize}

\end{frame}


\section{Objetivos}

\begin{frame}{Objetivos}

\begin{block}{Primer objetivo}
Caracterizar una serie de entornos, o casos de uso, a contemplar para la posterior elección de métricas, técnicas, y forma de abordar los problemas.
\end{block}

Tareas del primer objetivo:

\begin{description}
  \item[1.a] Identificar los entornos que se quieren cubrir y los problemas a resolver en cada uno.
  \item[1.b] Para cada entorno identificado, identificar también el contexto y los datos que se requieren para el análisis de cada uno.
\end{description}

\end{frame}

\begin{frame}{Objetivos}

\begin{block}{Segundo objetivo}
El segundo objetivo de la tesis es definir un conjunto de métricas en referencia a la detección de ataques de seguridad producidos por la insuficiencia o ineficacia de un conjunto de políticas establecido.
\end{block}

Tareas del segundo objetivo:

\begin{description}
  \item[2.a] Realizar un estudio sobre el estado del arte
    relacionado con los entornos identificados. 
  \item[2.b] Comparativa de métricas usadas en la literatura: usos, ventajas, e inconvenientes de cada una.
  \item[2.c] Definición de la métrica y justificación de la misma.
\end{description}

\end{frame}

\begin{frame}{Objetivos}

\begin{block}{Tercer objetivo}
Elegir qué técnicas de Soft Computing y de Machine Learning resultan las mejores para cada uno.
\end{block}

Tareas del tercer objetivo:

\begin{description}
  \begin{small}
  \item[3.a] Estudio del arte en técnicas Soft Computing y Machine Learning aplicadas a los entornos identificados. Este sub objetivo está relacionado con el primer objetivo y la tarea 2.a.
  \item[3.b] Relación entre la métrica escogida o definida en el segundo objetivo con las técnicas estudiadas en la tarea 3.a.
  \item[3.c] Justificación de las técnicas escogidas respecto a la métrica escogida o definida en el segundo objetivo.
  \end{small}
\end{description}

\end{frame}

\begin{frame}{Objetivos}

\begin{block}{Cuarto objetivo}
Proponer una metodología que defina qué datos procesar, y cómo procesarlos, para obtener la información necesaria para adaptar las reglas de seguridad existentes.
\end{block}

Tareas del cuarto objetivo:

\begin{description}
  \begin{small}
  \item[4.a] Estudio de las técnicas de pre-procesamiento y elección de una o varias, justificando el por qué. Se requiere este objetivo para el segundo paso de la metodología.
  \item[4.b] Creación de un \textit{benchmark} para la experimentación a partir de datos reales. Este objetivo afecta a los pasos 1 y 2 de la metodología.
  \end{small}
\end{description}

\end{frame}

\begin{frame}{Objetivos}

\begin{block}{Cuarto objetivo}
Proponer una metodología que defina qué datos procesar, y cómo procesarlos, para obtener la información necesaria para adaptar las reglas de seguridad existentes.
\end{block}

Tareas del cuarto objetivo:

\begin{description}
  \begin{small}
  \item[4.c] Diseñar y definir el entorno de procesamiento con las técnicas escogidas en el segundo objetivo de la tesis. Este sub objetivo es necesario para definir el segundo paso de la metodología.
  \item[4.d] Definir cómo se validan esos resultados: comparando las métricas de cada técnica entre sí, los porcentajes de acierto en clasificación, y extrayendo conocimiento a partir de los datos.
  \end{small}
\end{description}

\end{frame}

\begin{frame}{Metodología a proponer}
La metodología estaría compuesta de varios pasos que irían detallados con las conclusiones sacadas del cumplimiento de los objetivos anteriores:

\begin{enumerate}
  \item Un adecuado pre-procesamiento de datos para asegurar un buen mantenimiento de las bases de datos y máximo rendimiento. Descripción de cómo preparar los datos antes de analizarlos.
  \item Entorno de ejecución completo, es decir, arquitectura y técnicas usadas.
  \item Validación de los resultados con técnicas propuestas, modo de uso e interpretación.
\end{enumerate}

\end{frame}

\section{Metodología de trabajo}

\begin{frame}{Metodología de trabajo}

\end{frame}

\section{Plan de trabajo}

\begin{frame}{Plan de trabajo}
\includegraphics[width=0.95\textwidth]{./imgs/especiedegantt.png}
\end{frame}

\subsection{Medios y financiación}

\begin{frame}{Medios y financiación}
\$\$\$\$
\end{frame}

\end{document}
