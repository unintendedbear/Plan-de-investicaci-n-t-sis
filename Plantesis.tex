%% LyX 2.1.3 created this file.  For more info, see http://www.lyx.org/.
%% Do not edit unless you really know what you are doing.
%% This specific layout has been created by:
%% Antonio Fernández Ares
%% antares@ugr.es
%% Github @deantares
\documentclass[a4paper,spanish]{scrartcl}
\usepackage[T1]{fontenc}
\usepackage[utf8]{inputenc}
\usepackage{fancyhdr}
\pagestyle{fancy}
\usepackage{color}
\definecolor{note_fontcolor}{rgb}{1, 0.335938, 0}
\usepackage{array}
\usepackage{float}
\usepackage{endnotes}
\usepackage{graphicx}
\usepackage{setspace}
\onehalfspacing

\makeatletter

%%%%%%%%%%%%%%%%%%%%%%%%%%%%%% LyX specific LaTeX commands.
\pdfpageheight\paperheight
\pdfpagewidth\paperwidth

%% Because html converters don't know tabularnewline
\providecommand{\tabularnewline}{\\}
%% The greyedout annotation environment
\newenvironment{lyxgreyedout}
  {\textcolor{note_fontcolor}\bgroup\ignorespaces}
  {\ignorespacesafterend\egroup}

%%%%%%%%%%%%%%%%%%%%%%%%%%%%%% Textclass specific LaTeX commands.
 \let\footnote=\endnote

%%%%%%%%%%%%%%%%%%%%%%%%%%%%%% User specified LaTeX commands.
\usepackage{colortbl}

\definecolor{azulclaro}{rgb}{0.6,0.8,1}

%Aumentar el padding de las tablas
\usepackage{array}
\setlength\extrarowheight{3pt}

%Usar paquete xymatrix
\usepackage[all]{xy}


%Usar paquete para resaltado de código :)
\usepackage{listings}

%Usar paquete para gráficos
\usepackage{tikz}

\usepackage{pgfgantt}

%\newcommand{\paquete}[1]{\[\texttt{#1} \]}

\makeatother

\usepackage{babel}
\addto\shorthandsspanish{\spanishdeactivate{~<>}}

\begin{document}

\lhead{\includegraphics[height=1cm]{imgs/ugrLogo}}


\rhead{\includegraphics[height=1cm]{imgs/posgradoLogo!}}


\title{Plan Investigación}


\author{Paloma de las Cuevas Delgado}


\subtitle{\includegraphics[height=2cm]{imgs/ugrLogo}}

\maketitle
\newpage{}

\begin{center}
\begin{table}[H]
\centering{}%
\begin{tabular}{>{\raggedleft}m{5cm}>{\raggedright}m{8cm}}
\hline 
\multicolumn{2}{|c|}{\textcolor{cyan}{\cellcolor{azulclaro}}Plan de Investigación}\tabularnewline
\hline 
\hline 
\multicolumn{2}{|l|}{\textcolor{cyan}{\cellcolor{azulclaro}} Doctorando:}\tabularnewline
\hline 
Apellidos y nombre: & de las Cuevas Delgado, Paloma \tabularnewline
DNI: & 76418959X\tabularnewline
Correo electrónico: & palomacd@ugr.es\tabularnewline
Firma: & \tabularnewline
 & \tabularnewline
 & \tabularnewline
\hline 
Título provisional de la tesis: & Soft computing techniques applied to corporate and personal security\tabularnewline
Programa de doctorado: & DOCTORADO EN TECNOLOGÍAS DE LA INFORMACIÓN Y LA COMUNICACIÓN (B25.56.1) \tabularnewline
Línea de investigación: & Ingeniería Neuronal y Sistemas Integrados Bioinspirados\tabularnewline
\hline 
\multicolumn{2}{|l}{\textcolor{cyan}{\cellcolor{azulclaro}}Tutor/a:}\tabularnewline
\hline 
Apellidos y nombre: & Merelo Guervós, Juan Julián\tabularnewline
Correo electrónico: & jmerelo@ugr.es\tabularnewline
\hline 
\multicolumn{2}{|l}{\textcolor{cyan}{\cellcolor{azulclaro}}Director/a:}\tabularnewline
\hline 
Apellidos y nombre: & Merelo Guervós, Juan Julián\tabularnewline
Correo electrónico: & jmerelo@ugr.es\tabularnewline
\hline 
Apellidos y nombre: & García Sánchez, Pablo\tabularnewline
Correo electrónico: & pablogarcia@ugr.es \tabularnewline
\hline 
\end{tabular}
\end{table}

\par\end{center}

\newpage{}


\part{Memoria de Investigación}


\section{Título (provisional) de la tesis}

La tesis tiene por título ``Soft computing techniques applied to corporate and personal security'', 


\section{Introducción}

La evolución desde los teléfonos móviles que se pueden llamar ``tradicionales'' (primeras generaciones), hasta lo que se conoce hoy en día como ``smartphones'', ha cambiado la manera en la que las personas utilizan sus dispositivos móviles. De la misma manera, las amenazas de seguridad también han evolucionado \cite{gangula2013survey}, y como consecuencia se deben tomar nuevas medidas de seguridad a medida que aparecen estas amenazas.
En concreto, los ``smartphones'' han propiciado el desarrollo, en el entorno de oficina, de lo que se ha denominado como un escenario \textit{Bring Your Own Device} (BYOD), que se podría traducir como ``Traiga su propio dispositivo''. Esta práctica consiste en que una determinada empresa permite a sus empleados el uso de sus dispositivos móviles personales para trabajar, en lugar de otros que sean propiedad de la empresa. Esto significa que los empleados de la compañía pueden usar los dispositivos para cosas personales en el puesto de trabajo, a la vez que realizan tareas del trabajo fuera del mismo. A pesar de la gran cantidad de ventajas que esto pueda suponer, es claro que este tipo de situación crea nuevos retos de seguridad para el jefe del departamento de seguridad (en inglés Chief Security Officer o CSO) de la compañía \cite{Opp_Security11}. Esto se debe a que la compañía necesita una respuesta lo suficientemente rápida ante cualquier comportamiento por parte de los usuarios que pueda causar pérdidas de dinero. Sin embargo, cualquier medida de seguridad que la compañía quiera tomar deben evitar monitorizar los dispositivos de los empleados de una manera que atente contra su privacidad.
De este modo, la tarea de un CSO y en general del departamento de seguridad dentro de una compañía es: establecer una serie de medidas de seguridad para hacer frente a todos los incidentes de seguridad, construyendo así un conjunto de ``políticas de seguridad corporativas'' (en inglés \textit{Corporate Security Policies} o CSPs). Estas políticas consisten en un conjunto de reglas de seguridad enfocadas a la protección de los activos de la empresa a través de la definición de permisos para comportamientos específicos que pudieran provocar incidentes de seguridad \cite{kaeo2003designing}.
Sin embargo, si las compañías deciden adoptar un escenario tan cambiante como lo es el de BYOD, y permitir a sus empleados utilizar sus propios dispositivos en el trabajo o para trabajar, el riesgo de ocurrencia de incidentes de seguridad crece inminentemente, a pesar de que los empleados no tengan un deseo explícito de perjudicar a la empresa \cite{stanton2005analysis, breivik2002abstract}. Por tanto, se crea la necesidad de renovar o adaptar, de manera dinámica, el conjunto de políticas de seguridad corporativas o CSPs.

En la tésis se propone una arquitectura, que sea fácilmente integrable en los servidores de una compañía, y que sea capaz de evolucionar las reglas incluídas en un conjunto de políticas de seguridad corporativas. El sistema alcanzaría este objetivo a través del aprendizaje del comportamiento que tuviesen los empleados o usuarios en el pasado y que supusiesen un incidente de seguridad. De este modo, el sistema tendría en cuenta las violaciones de seguridad conocidas porque han sido contempladas en las políticas de seguridad, pero también otros aspectos del comportamiento de los usuarios, observación del entorno, y valor de los activos, entre otros. Para ello, se van a aplicar diferentes técnicas.

En primer lugar, se asume que las compañías almacenan la información sobre los incidentes de seguridad, junto con su \textit{contexto}. El término \textit{contexto} fue definido por Abowd et al. en \cite{abowd1999towards} como\footnote{En inglés en el artículo del autor} "cualquier información que pueda usarse para caracterizar la situación de una entidad". Así, y dado que esto puede significar el análisis de grandes cantidades de datos, es necesario emplear técnicas de minería de datos para la extracción de información de los datos de los que se parte \cite{DeVel2001}, ya sea de comportamientos no peligrosos como de los que causen incidentes de seguridad. Este proceso permitiría la construcción o entrenamiento de un clasificador para obtener dos resultados: primero, para poder usar el modelo del clasificador como posibles reglas a contrastar con las existentes, y segundo, para poder clasificar futuras situaciones o acciones de los usuarios. Por último, puesto que las reglas tienen estructura de árbol, donde las ramas serían las condiciones con los distintos nodos los valores, y las hojas las decisiones, se podrían aplicar Algoritmos Evolutivos (AEs) para optimizar su estructura \cite{de2002discovering}.

\subsection{Proyecto MUSES}


\section{Hipótesis y justificación}


\section{Objetivos}

\subsection{Primer Objetivo}

Definir qué métricas son más interesantes. Esto permitirá que...

\subsection{Segundo Objetivo}

Comparar qué técnicas de Soft Computing son más adecuadas para un entorno como este.

\subsection{Tercer Objetivo}

Desarrollar una metodología (que se pueda vender a la gente) compuesta de varios pasos:
\begin{enumerate}
  \item Pre-procesamiento de datos
  \item Ejecución, comparativa y análisis
  \item Validación de los resultados
\end{enumerate}


\section{Metodología}



\section{Plan de trabajo}

% Dejo esto aquí comentado para aprender cómo se hacen los diagramas de Gantt en LaTeX
%
% \definecolor{barblue}{RGB}{153,204,254}
% \definecolor{groupblue}{RGB}{51,102,254}
% \definecolor{linkred}{RGB}{165,0,33}
% \renewcommand\sfdefault{phv}
% \renewcommand\mddefault{mc}
% \renewcommand\bfdefault{bc}
% \setganttlinklabel{s-s}{EMPEZAR-PARA-EMPEZAR}
% \setganttlinklabel{f-s}{TERMINAR-PARA-EMPEZAR}
% \setganttlinklabel{f-f}{TERMINAR-PARA-TERMINAR}
% \sffamily
% 
% \begin{ganttchart}[
% canvas/.append style={fill=none, draw=black!5, line width=.75pt},
% hgrid style/.style={draw=black!5, line width=.75pt},
% vgrid={*1{draw=black!5, line width=.75pt}},
% today=7,
% today rule/.style={
% draw=black!64,
% dash pattern=on 3.5pt off 4.5pt,
% line width=1.5pt
% },
% today label font=\small\bfseries,
% title/.style={draw=none, fill=none},
% title label font=\bfseries\footnotesize,
% title label node/.append style={below=7pt},
% include title in canvas=false,
% bar label font=\mdseries\small\color{black!70},
% bar label node/.append style={left=2cm},
% bar/.append style={draw=none, fill=black!63},
% bar incomplete/.append style={fill=barblue},
% bar progress label font=\mdseries\footnotesize\color{black!70},
% group incomplete/.append style={fill=groupblue},
% group left shift=0,
% group right shift=0,
% group height=.5,
% group peaks tip position=0,
% group label node/.append style={left=.6cm},
% group progress label font=\bfseries\small,
% link/.style={-latex, line width=1.5pt, linkred},
% link label font=\scriptsize\bfseries,
% link label node/.append style={below left=-2pt and 0pt}
% ]{1}{36}
% \gantttitle[
% title label node/.append style={below left=7pt and -3pt}
% ]{MESES:\quad1}{1}
% \gantttitlelist{2,...,12}{1} \\
% \ganttgroup[progress=57]{\txt{Dispositivo}}{1}{10} \\
% \ganttbar[
% progress=75,
% name=WBS1A
% ]{\textbf{WBS 1.1} Activity A}{1}{8} \\
% \ganttbar[
% progress=67,
% name=WBS1B
% ]{\textbf{WBS 1.2} Activity B}{1}{3} \\
% \ganttbar[
% progress=50,
% name=WBS1C
% ]{\textbf{WBS 1.3} Activity C}{4}{10} \\
% \ganttbar[
% progress=0,
% name=WBS1D
% ]{\textbf{WBS 1.4} Activity D}{4}{10} \\[grid]
% \ganttgroup[progress=0]{WBS 2 Summary Element 2}{4}{10} \\
% \ganttbar[progress=0]{\textbf{WBS 2.1} Activity E}{4}{5} \\
% \ganttbar[progress=0]{\textbf{WBS 2.2} Activity F}{6}{8} \\
% \ganttbar[progress=0]{\textbf{WBS 2.3} Activity G}{9}{10}
% \ganttlink[link type=s-s]{WBS1A}{WBS1B}
% \ganttlink[link type=f-s]{WBS1B}{WBS1C}
% \ganttlink[
% link type=f-f,
% link label node/.append style=left
% ]{WBS1C}{WBS1D}
% \end{ganttchart}


\section{Medios y financiación}


\part{Anexos}

\bibliographystyle{abbrv}
\bibliography{PlanTesis}

\end{document}
