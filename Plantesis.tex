%% LyX 2.1.3 created this file.  For more info, see http://www.lyx.org/.
%% Do not edit unless you really know what you are doing.
%% This specific layout has been created by:
%% Antonio Fernández Ares
%% antares@ugr.es
%% Github @deantares
\documentclass[a4paper,spanish]{scrartcl}
\usepackage[T1]{fontenc}
\usepackage[utf8]{inputenc}
\usepackage{fancyhdr}
\pagestyle{fancy}
\usepackage{color}
\definecolor{note_fontcolor}{rgb}{1, 0.335938, 0}
\usepackage{array}
\usepackage{float}
\usepackage{endnotes}
\usepackage{graphicx}
\usepackage{setspace}
\onehalfspacing

\makeatletter

%%%%%%%%%%%%%%%%%%%%%%%%%%%%%% LyX specific LaTeX commands.
\pdfpageheight\paperheight
\pdfpagewidth\paperwidth

%% Because html converters don't know tabularnewline
\providecommand{\tabularnewline}{\\}
%% The greyedout annotation environment
\newenvironment{lyxgreyedout}
  {\textcolor{note_fontcolor}\bgroup\ignorespaces}
  {\ignorespacesafterend\egroup}

%%%%%%%%%%%%%%%%%%%%%%%%%%%%%% Textclass specific LaTeX commands.
 \let\footnote=\endnote

%%%%%%%%%%%%%%%%%%%%%%%%%%%%%% User specified LaTeX commands.
\usepackage{colortbl}

\definecolor{azulclaro}{rgb}{0.6,0.8,1}

%Aumentar el padding de las tablas
\usepackage{array}
\setlength\extrarowheight{3pt}

%Usar paquete xymatrix
\usepackage[all]{xy}


%Usar paquete para resaltado de código :)
\usepackage{listings}

%Usar paquete para gráficos
\usepackage{tikz}

\usepackage{pgfgantt}

%\newcommand{\paquete}[1]{\[\texttt{#1} \]}

\makeatother

\usepackage{babel}
\addto\shorthandsspanish{\spanishdeactivate{~<>}}

\begin{document}

\lhead{\includegraphics[height=1cm]{imgs/ugrLogo}}


\rhead{\includegraphics[height=1cm]{imgs/posgradoLogo!}}


\title{Plan Investigación}


\author{Paloma de las Cuevas Delgado}


\subtitle{\includegraphics[height=2cm]{imgs/ugrLogo}}

\maketitle
\newpage{}

\begin{center}
\begin{table}[H]
\centering{}%
\begin{tabular}{>{\raggedleft}m{5cm}>{\raggedright}m{8cm}}
\hline 
\multicolumn{2}{|c|}{\textcolor{cyan}{\cellcolor{azulclaro}}Plan de Investigación}\tabularnewline
\hline 
\hline 
\multicolumn{2}{|l|}{\textcolor{cyan}{\cellcolor{azulclaro}} Doctorando:}\tabularnewline
\hline 
Apellidos y nombre: & de las Cuevas Delgado, Paloma \tabularnewline
DNI: & 76418959X\tabularnewline
Correo electrónico: & palomacd@ugr.es\tabularnewline
Firma: & \tabularnewline
 & \tabularnewline
 & \tabularnewline
\hline 
Título provisional de la tesis: & Soft computing techniques applied to corporate and personal security\tabularnewline
Programa de doctorado: & DOCTORADO EN TECNOLOGÍAS DE LA INFORMACIÓN Y LA COMUNICACIÓN (B25.56.1) \tabularnewline
Línea de investigación: & Ingeniería Neuronal y Sistemas Integrados Bioinspirados\tabularnewline
\hline 
\multicolumn{2}{|l}{\textcolor{cyan}{\cellcolor{azulclaro}}Tutor/a:}\tabularnewline
\hline 
Apellidos y nombre: & Merelo Guervós, Juan Julián\tabularnewline
Correo electrónico: & jmerelo@ugr.es\tabularnewline
\hline 
\multicolumn{2}{|l}{\textcolor{cyan}{\cellcolor{azulclaro}}Director/a:}\tabularnewline
\hline 
Apellidos y nombre: & Merelo Guervós, Juan Julián\tabularnewline
Correo electrónico: & jmerelo@ugr.es\tabularnewline
\hline 
Apellidos y nombre: & García Sánchez, Pablo\tabularnewline
Correo electrónico: & pablogarcia@ugr.es \tabularnewline
\hline 
\end{tabular}
\end{table}

\par\end{center}

\newpage{}


\part{Memoria de Investigación}


\section{Título (provisional) de la tesis}

La tesis tiene por título ``Soft computing techniques applied to
corporate and personal security'', en español ``Aplicación de técnicas
de Soft Computing a la seguridad corporativa y personal'' o ``Sistema
para la autoadaptación de reglas aplicadas a seguridad corporativa y
personal''. 


\section{Hipótesis}

% Antares's halp:
% La hipótesis es "que pregunta quieres responder tu tesis", los objetivos es "que pretendes lograr para responerlo" y la metología es "como pretendes hacerlo"

La evolución desde los teléfonos móviles que se pueden llamar
``tradicionales'' o de primeras generaciones, hasta lo que se conoce hoy
% los paréntesis entorpecen la lectura, úsalos sólo de forma muy
% justificada - JJ
en día como ``smartphones'', ha cambiado la manera en la que las
personas utilizan estos dispositivos. %FERGU: dispositivos
                                %móviles->estos dispositivos. 
De la misma manera, las amenazas de seguridad también han evolucionado \cite{gangula2013survey}, y como consecuencia se deben tomar nuevas medidas de seguridad a medida que aparecen estas amenazas.
En concreto, los ``smartphones'' han propiciado el desarrollo, en el entorno de oficina, de lo que se ha denominado como un escenario \textit{Bring Your Own Device} (BYOD), que se podría traducir como ``Traiga su propio dispositivo''. Esta práctica consiste en que una determinada empresa permite a sus empleados el uso de sus dispositivos móviles personales para trabajar, en lugar de otros que sean propiedad de la empresa. Esto significa que los empleados de la compañía pueden usar los dispositivos para uso personal %FERGU: uso personal
en el puesto de trabajo, a la vez que realizan tareas del trabajo fuera del mismo. A pesar de la gran cantidad de ventajas que esto pueda suponer, es claro que este tipo de situación crea nuevos retos de seguridad para el jefe del departamento de seguridad (en inglés Chief Security Officer o CSO) de una empresa \cite{Opp_Security11}. %FERGU: de una compañía, en vez de la compañía. O "de las compañías". Usar también la palabra "empresa" si suena muy repetitivo
Esto se debe a que la compañía necesita una respuesta lo suficientemente rápida ante cualquier comportamiento por parte de los usuarios que pueda causar pérdidas de dinero. Sin embargo, cualquier medida de seguridad que la compañía quiera tomar deben evitar monitorizar los dispositivos de los empleados de una manera que atente contra su privacidad.
De este modo, la tarea de un CSO, y en general del departamento de seguridad dentro de una compañía es establecer una serie de medidas de seguridad para hacer frente a todos los incidentes de seguridad, construyendo así un conjunto de ``políticas de seguridad corporativas'' (en inglés \textit{Corporate Security Policies} o CSPs). %FERGU: no pondría los dos puntos si no vas a hacer una enumeración
 Estas políticas consisten en un conjunto de reglas de seguridad enfocadas a la protección de los activos de la empresa a través de la definición de permisos para comportamientos específicos que pudieran provocar incidentes de seguridad \cite{kaeo2003designing}.
Sin embargo, si las compañías deciden adoptar un escenario tan cambiante como lo es el de BYOD, y permitir a sus empleados utilizar sus propios dispositivos en el trabajo o para trabajar, el riesgo de ocurrencia de incidentes de seguridad crece inminentemente, a pesar de que los empleados no tengan un deseo explícito de perjudicar a la empresa \cite{stanton2005analysis, breivik2002abstract}. Por tanto, se crea la necesidad de renovar o adaptar, de manera dinámica, el conjunto de CSPs. %FERGU: ya has definido CSPs antes, deja sólo el acrónimo

%FERGU: yo empezaría el siguiente párrafo así: Por lo tanto, el objetivo de esta tesis es demostrar que es posible desarrollar una arquitectura, que sea...
Por lo tanto, el objetivo de esta tesis es demostrar que es posible desarrollar una arquitectura que sea fácilmente integrable en los servidores de una compañía, y que sea capaz de evolucionar las reglas incluídas en un conjunto de políticas de seguridad corporativas. El sistema alcanzaría este objetivo a través del aprendizaje del comportamiento que tuviesen los empleados o usuarios en el pasado y que supusiesen un incidente de seguridad. De este modo, el sistema tendría en cuenta las violaciones de seguridad conocidas porque han sido contempladas en las políticas de seguridad, pero también otros aspectos del comportamiento de los usuarios, observación del entorno, y valor de los activos, entre otros. Para ello, se van a aplicar diferentes técnicas.

Para resolver el problema, se asume que las compañías almacenan la
información sobre los incidentes de seguridad, junto con su
\textit{contexto}. El término \textit{contexto} fue definido por Abowd
et al. en \cite{abowd1999towards} como\footnote{En inglés en el
  artículo del autor}:
  
 \begin{verbatim}
    Cualquier información que pueda usarse para caracterizar la situación 
    de una entidad.
  \end{verbatim} % sería mejor que usaras
                                % \verbatim o similar
 Así, y dado que esto puede significar el análisis de grandes cantidades de datos, es necesario emplear técnicas de minería de datos para la extracción de información de los datos de los que se parte \cite{DeVel2001}, ya sea de comportamientos no peligrosos como de los que causen incidentes de seguridad. Este proceso permitiría la construcción o entrenamiento de un clasificador para obtener dos resultados: primero, para poder usar el modelo del clasificador como posibles reglas a contrastar con las existentes, y segundo, para poder clasificar futuras situaciones o acciones de los usuarios. Por último, puesto que las reglas tienen estructura de árbol, donde las ramas serían las condiciones con los distintos nodos los valores, y las hojas las decisiones, se podrían aplicar Algoritmos Evolutivos (AEs) para optimizar su estructura \cite{de2002discovering}.

\subsection{Antecedentes: Herramientas actuales y Técnicas de Soft Computing}

En los últimos cuatro años se han desarrollado varias herramientas, enfocadas tanto a empresas como a usuarios particulares, destinadas a ayudar a la adaptación a entornos BYOD. Así, teniendo como objetivo el mundo de la empresa, existen herramientas de compañías como IBM, IBM's Hosted Mobile Device Security Management, y Sophos, que ofrece Sophos Mobile Control. Ambas herramientas ofrecen distintas maneras al CSO de controlar los dispositivos que se registran en el sistema, pudiendo éste requerir a los usuarios el uso de contraseñas más seguras, por ejemplo, o siendo posible proteger los datos de los usuarios encriptando la base de datos.
Otra herramienta con características a las dos anteriores es la que ofrece Good's Technology, con la diferencia de que además incorpora una serie de directrices para una elaboración correcta y efectiva de olíticas y reglas de seguridad.
Sin embargo, ninguna de estas tres herramientas incorpora la habilidad de crear reglas de seguridad nuevas ni de adaptar las existentes como se propone en esta tesis.

En cuanto a los dispositivos, parece lógica la solución adoptada por Silent Circle: la creación de un smartphone con el único objetivo de proteger los datos que tiene en él, llamado ``Blackphone'' \cite{Blackphone_site}. Este dispositivo incorpora un sistema operativo llamado \textit{PrivatOS} y que está basado en Android. Incluso tiene su propia tienda de aplicaciones, llamada \textit{Silent Store} y cuyo objetivo es maximizar la privacidad y la seguridad, puesto que los permisos que se le puedan conceder a determinadas aplicaciones durante la instalación puede dar lugar más tarde a filtración de datos personales \cite{gangula2013survey}. Otra de las características de este dispositivo es el borrado remoto, en el caso de que haya sido sustraído o se haya perdido.

Esta solución tiene una clara desventaja, que puede formularse de dos maneras: o bien la empresa realiza una gran inversión para proporcionar estos terminales a sus empleados, lo cual va en contra de lo que sería la filosofía BYOD, o requieren que los empleados se lo compren, de modo que tampoco podrían usar los dispositivos que ya posean. Por contra, el sistema que se propone desarrollar durante la tesis está diseñado para ser independiente de la plataforma.

Finalmente, y pudiendo ser consideradas como una extensión para móviles con sistema operativo Android, aparecen dos herramientas en el mercado: Samsung KNOX para dispositivos de este fabricante, y Android Work desarrollado por el propio Google. Ambas herramientas tienen las mismas ventajas que el Blackphone ya mencionado, incorporando además sendas aplicaciones de servidor para los CSO. Esto significa que tanto Samsung como Google ofrecen alternativas para ambos cliente y servidor. Es más, Android Work sigue la filosofía que ya comenzó a verse en los teléfonos de la marca Blackberry, con la herramienta Blackberry Balance, la cual consiste en tener por separado aplicaciones para el trabajo y personales. A los smartphones que adoptan esta práctica se les llama de ``dual-persona'' \cite{AndroidWork_review}. Sin embargo, en cuanto a políticas de seguridad, ninguna de las dos herramientas de Samsung o Google especifican la autoadaptación como característica. Sí ofrecen gestión de políticas existentes y aplicación de las mismas, pero aún así no parece que hagan un análisis de los datos del sistema con el objetivo de evolucionar un conjunto de reglas de seguridad inicial.

Un primer estudio sobre las herramientas existentes ha sido publicado en \cite{de2015corporate}.

% 
% With respect to the application of Data Mining to extract information from big amounts of data, this has been done since the nineties \cite{agrawal1995mining, ester1996density}. More specifically, DM has been widely used for security purposes, as it can be applied in computer forensics. O. de Vel studied the application of DM techniques to identify authors of malicious e-mails in \cite{DeVel2001}, and for performing ``offender profiling'' in relation to computer security attacks in \cite{abraham2002investigative}. Yet, the system this paper proposed is focused in doing this kind of analysis but then to look for similarities with the new incoming events, so that a decision can be made in case they are dangerous. Classification methods are also applied in the security field. For instance, Blanzieri and Bryl \cite{blanzieri2008} present a review on a variety of spam filtering methods, and compare them, reaching the conclusion of that they are successful in general, but yet insufficient. This is why implementing a self-adaptive system such as the one this paper proposes can be good for other security applications and not only spam classification.
% 
% As for the works related with the users' information and behaviour, and the management (and adaptation) of the set of Corporate Security Policies, many can be found in literature. For instance, P.G. Kelley et al. \cite{user-controllable_learning_08} presented a method named \textit{user-controllable policy learning} in which the user gives feedback to the system every time it applies a security policy. Then, these policies can be refined according to that feedback to be more accurate with respect to what the users need. This approach could be useful for adding information to the system, and therefore perform a deeper analysis to extract more accurate conclusions, and finally create better rules.
% Then, taking into account how much information can be gathered from social networks, Danezis in \cite{inferring_policies_socialnetworks_09} defined a system able to infer privacy-related restrictions, enhancing user's privacy, by applying Machine Learning techniques on a social network environment. Again, this is another interesting approach. However, this paper focuses on CSPs, related to companies, more than on personal life of individuals.
% 
% In the same line, Lim et al. proposed a system \cite{lim2008mls, lim2008policy} which evolves a set of computer security policies by means of Genetic Programming, gathering knowledge from the user's feedback like in \cite{user-controllable_learning_08}. Furthermore, Suarez-Tangil et al. \cite{suarez2009automatic} take the same approach as Lim et al., but also including event correlation in. These two latter author's works are interesting for this paper, though they are not focused on company CSPs.

\section{Justificación}

La tesis continúa el trabajo comenzado durante el Proyecto europeo MUSES \footnote{https://www.musesproject.eu/} y el Trabajo Fin de Máster \cite{palomaTFM2014}, cuyos resultados han sido además publicados en \cite{mora14:urls}.

En este primer estudio realizado durante el Trabajo Fin de Máster, se trabajó con un conjunto de datos reales y compuestos por una serie de peticiones HTTP que se realizaban desde una oficina. Los atributos que se obtenían de cada petición eran:

\begin{itemize}
  \item \texttt{http\_reply\_code}, el código de respuesta del servidor HTTP.
  \item \texttt{http\_method}, la acción que se desea hacer sobre el recurso al que se quiere acceder.
  \item \texttt{duration\_milliseconds}, tiempo que transcurre desde que se hace la petición al servidor hasta que éste responde.
  \item \texttt{content\_type\_MCT}, el tipo de recurso principal al que se quiere acceder.
  \item \texttt{content\_type}, el tipo de recurso específico al que se quiere acceder.
  \item \texttt{server\_or\_cache\_address}, la dirección IP del servidor.
  \item \texttt{time}, hora a la que se produce la petición.
  \item \texttt{squid\_hierarchy}, indica si el siguiente salto es directamente al servidor o al proxy (Squid).
  \item \texttt{bytes}, cantidad de información transferida.
  \item \texttt{client\_address}, dirección IP del cliente. Esta característica se eliminó a posteriori, para evitar que se creasen reglas que filtrasen por cliente.
  \item \texttt{URL}, el servidor principal al que se accede. Por ejemplo, \textit{ugr} en \textit{http://www.ugr.es}.
\end{itemize}

Cabe destacar el uso de métodos de clasificación que admitiesen tanto variables numéricas (\texttt{duration\_milliseconds}) como categóricas (\texttt{http\_method}), siendo de este tipo la mayoría.

\section{Objetivos}

\subsection{Primer Objetivo}

Definir qué métricas son más interesantes. Esto permitirá que...

\subsection{Segundo Objetivo}

Comparar qué técnicas de Soft Computing son más adecuadas para un entorno como este.

\subsection{Tercer Objetivo}

Desarrollar una metodología (que se pueda vender a la gente) compuesta de varios pasos:
\begin{enumerate}
  \item Pre-procesamiento de datos
  \item Ejecución, comparativa y análisis
  \item Validación de los resultados
\end{enumerate}


\section{Metodología}

El sistema que se propone desarrollar durante la tesis está pensado para ser instalado en el servidor de una compañía que desee añadir el proceso de refinamiento de reglas de seguridad. Este sistema puede incluso ser propuesto como extensión a alguna de las herramientas que ya existen en el mercado y que han sido comentadas en los antecedentes. La Figura \ref{fig:krs} muestra un resumen de los componentes de la arquitectura dentro del sistema que se propone.

En cuanto al flujo de información entre componentes, cabe destacar que la \textit{base de datos} se refiere a la parte de información que la compañía retiene acerca de los eventos o acciones que se reciben por parte de los usuarios, de manera anonimizada. Por tanto, el sistema contribuye a la conservación de la privacidad, puesto que sólo accede a parte de la base de datos de la compañía. A continuación, se detallan los dos componentes principales: el componente donde se realiza la \textit{Minería de Datos}, y el componente que realiza el \textit{Tratamiento de Reglas}, creándolas o adaptándolas mediante Algoritmos Evolutivos.

\begin{figure*}
  \begin{center}
    \includegraphics[width=0.75\textwidth]{./imgs/KRS.png}
    \caption{Arquitectura a nivel de componentes y subcomponentes del sistema propuesto.}
    \label{fig:krs}
  \end{center}
\end{figure*}

\subsection{Minería de datos}
\label{subsec:datamining}

Este componente se encargará de consultar la base de datos, obtener la información que se considere relevante, y procesarla para borrar entradas que contengan errores o valores no válidos. Por ejemplo, las entradas en la base de datos que estén repetidas o que contengan muchos valores que no se han podido monitorizar son susceptibles de ser borradas en este proceso. Los datos de interés son los correspondientes a los eventos o acciones, junto con su contexto, producidos por los usuarios cuando interactúan con el sistema. El resultado de este proceso es un conjunto de datos formado por lo que se llaman ``patrones'' \cite{duran2007mineria} (en inglés \textit{pattern}), que a su vez son una serie de valores para los atributos o variables del evento que se han considerado importantes. En la fase de preprocesamiento, también se pueden añadir atributos a partir del estudio de las variables iniciales. Por ejemplo: se considera de interés la contraseña de acceso al sistema del usuario, pero no se toma la contraseña en sí con el fin de preservar su privacidad; en su lugar, se obtienen atributos como \textit{el número de caracteres en total}, \textit{porcentaje de números y letras}, o \textit{cantidad de letras mayúsculas}.
Por tanto, el componente de preprocesamiento será el encargado de preparar los datos para la posterior aplicación de técnicas como la minería de patrones \cite{han2007frequent}. Esta técnica permite la identificación de patrones poco frecuentes o anómalos ya que estos son, en principio, sospechosos y por tanto de interés para el departamento de seguridad.

El siguiente subcomponente se encarga de tareas como la selección de características \cite{guyon2003introduction}, la cual es una técnica que consiste en la elección de un subconjunto de variables o atributos (obtenidos de los eventos en el paso anterior) para poder reducir el tamaño del conjunto de datos. Este paso se realiza con el fin de mejorar el rendimiento del sistema y, en general, del proceso de clasificación, ya que la velocidad de procesamiento aumenta sin que se pierda porcentaje de acierto al clasificar.
Seguidamente, este subcomponente aplica una serie de algoritmos de clasificación \cite{witten2005data}, esto es, usa los datos preprocesados para entrenar un modelo (o clasificador) capaz de asociar cada patrón del conjunto de datos con una clase (o etiqueta). Así, el clasificador puede asignar una clase, o en nuestro caso una ``decisión'', a futuros eventos.
Además, los algoritmos de clasificación utilizados serán basados en reglas o árboles, para poder utilizar el clasificador entrenado por los datos obtenidos como conjunto de reglas nuevas, que será la entrada para el siguiente componente.

\subsection{Tratamiento de reglas}
\label{subsec:ruletreatment}

Este módulo estará enfocado en la creación de nuevas reglas, además de trabajar con el conjunto de políticas existente. De este modo, realizará tres acciones diferentes e independientes sobre las reglas:

\begin{itemize}
  \item Primero, tomará el conjunto de reglas que se han obtenido de entrenar el clasificador en el paso anterior y las comparará con las existentes para comprobar si se ha obtenido alguna regla nueva. Este proceso se realiza una vez al día, cada día tomando todo el histórico de datos más los datos nuevos obtenidos.
  \item Después, se hace una recomprobación de todas las reglas para evitar redundancias o reglas que se contradigan entre sí, manteniendo la coherencia en el conjunto de reglas.
  \item Finalmente, aprovechando la estructura de árbol que tiene una regla, el sistema utilizará Programación Genética \cite{koza1992genetic}, de entre los Algoritmos Genéticos existentes, para optimizar estructuras basadas en árboles. Esto también contribuye a la comprensión de las nuevas reglas por parte del CSO (jefe del departamento de seguridad) \cite{tan2002mining}. De esta manera, el conjunto final de reglas se presentará al CSO, el cual podrá aceptarlas o rechazarlas, repercutiendo esta decisión a su vez en el sistema.
\end{itemize}

\section{Plan de trabajo}

% Dejo esto aquí comentado para aprender cómo se hacen los diagramas de Gantt en LaTeX
%
% \definecolor{barblue}{RGB}{153,204,254}
% \definecolor{groupblue}{RGB}{51,102,254}
% \definecolor{linkred}{RGB}{165,0,33}
% \renewcommand\sfdefault{phv}
% \renewcommand\mddefault{mc}
% \renewcommand\bfdefault{bc}
% \setganttlinklabel{s-s}{EMPEZAR-PARA-EMPEZAR}
% \setganttlinklabel{f-s}{TERMINAR-PARA-EMPEZAR}
% \setganttlinklabel{f-f}{TERMINAR-PARA-TERMINAR}
% \sffamily
% 
% \begin{ganttchart}[
% canvas/.append style={fill=none, draw=black!5, line width=.75pt},
% hgrid style/.style={draw=black!5, line width=.75pt},
% vgrid={*1{draw=black!5, line width=.75pt}},
% today=7,
% today rule/.style={
% draw=black!64,
% dash pattern=on 3.5pt off 4.5pt,
% line width=1.5pt
% },
% today label font=\small\bfseries,
% title/.style={draw=none, fill=none},
% title label font=\bfseries\footnotesize,
% title label node/.append style={below=7pt},
% include title in canvas=false,
% bar label font=\mdseries\small\color{black!70},
% bar label node/.append style={left=2cm},
% bar/.append style={draw=none, fill=black!63},
% bar incomplete/.append style={fill=barblue},
% bar progress label font=\mdseries\footnotesize\color{black!70},
% group incomplete/.append style={fill=groupblue},
% group left shift=0,
% group right shift=0,
% group height=.5,
% group peaks tip position=0,
% group label node/.append style={left=.6cm},
% group progress label font=\bfseries\small,
% link/.style={-latex, line width=1.5pt, linkred},
% link label font=\scriptsize\bfseries,
% link label node/.append style={below left=-2pt and 0pt}
% ]{1}{36}
% \gantttitle[
% title label node/.append style={below left=7pt and -3pt}
% ]{MESES:\quad1}{1}
% \gantttitlelist{2,...,12}{1} \\
% \ganttgroup[progress=57]{\txt{Dispositivo}}{1}{10} \\
% \ganttbar[
% progress=75,
% name=WBS1A
% ]{\textbf{WBS 1.1} Activity A}{1}{8} \\
% \ganttbar[
% progress=67,
% name=WBS1B
% ]{\textbf{WBS 1.2} Activity B}{1}{3} \\
% \ganttbar[
% progress=50,
% name=WBS1C
% ]{\textbf{WBS 1.3} Activity C}{4}{10} \\
% \ganttbar[
% progress=0,
% name=WBS1D
% ]{\textbf{WBS 1.4} Activity D}{4}{10} \\[grid]
% \ganttgroup[progress=0]{WBS 2 Summary Element 2}{4}{10} \\
% \ganttbar[progress=0]{\textbf{WBS 2.1} Activity E}{4}{5} \\
% \ganttbar[progress=0]{\textbf{WBS 2.2} Activity F}{6}{8} \\
% \ganttbar[progress=0]{\textbf{WBS 2.3} Activity G}{9}{10}
% \ganttlink[link type=s-s]{WBS1A}{WBS1B}
% \ganttlink[link type=f-s]{WBS1B}{WBS1C}
% \ganttlink[
% link type=f-f,
% link label node/.append style=left
% ]{WBS1C}{WBS1D}
% \end{ganttchart}


\section{Medios y financiación}


\part{Anexos}

\bibliographystyle{abbrv}
\bibliography{PlanTesis}

\end{document}
